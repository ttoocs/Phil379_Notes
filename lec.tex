%Jan 10, 2017.

\subsection{Misc. Notation}

\begin{itemize}
\item The set of positive integers $\{ x : x$ is a positive integer $\}$ 
\item The set of positive integers less than3 $\{ x : x $ is a positive integer and x is less than 3 $\}$. $= \{1,2\}$. 
%I'm already bored.
\item The empty set: $\emptyset \tor \Delta$ 
\item Member of: $A \subseteq B$ iff $\forall X (x \in A \implies x \in B )$ 
\item Union of A and B: $A \cup B $ iff $\{ x : x \in A \lor x \in B \}$
\item Intersection of A and B: $A \cap B $ iff $\{ x : x \in A \land x \in B \}$
\item Difference of A and B: $\{ x : x \in A \land x \not \in B \}$
\item For any non-empty sets $A,B$:
Cartesian product: A of B: $A \cross B$: $\{<x,y>: x \in A \land y \in B \}$ (ALL OF THE POSSIBILITIES)
\item TOTAL FUNCTION: Every element in the domain is valid
\item PARTIAL FUNCTION: Not every element in the domain is valid.
%%LEC02:
\item for any set of sets $A$:
\begin{itemize}
\item $\Cup A = \{ x : \exists y ( y \in A \land x \in y ) \}$ 
\item $\Cap A = \{ x : \forall y (y \in A \to x \in y )\}$ 
\end{itemize}
%%OHBOYTHESEAGAIN
\item Relations: R is 
\begin{itemize}
\item reflexive : $\forall x Rxx$
\item symmetric : $\forall x \forall y (Rxy \implies Ryx)$
\item transitive : $\forall x \forall y \forall z ((Rxy \land Ryz) \implies Rxz)$
\item Euclidean : $\forall x \forall y \forall z ((Rxy \land Rxz ) \implies Ryz)$
\item a equivalence relation : it's symmetric,reflexive,transitive.
\item a equivalence relation (alt) : it's symmetric, and euclidean.
\item a (partial) function : $\exists x$ and there is at most one y: $Rxy$ : denoted $f$
\item a (partial) function$^{1}$ :  $\exists x , \exists y | Rxy$ : denoted $f$.
\item a (total) function: assigns a value to each number of $A$ : denoted $f$
%\item a inverse function: if $\forall x $...?
\item a (total) function$^{2}$: $\forall x , \exists y | Rxy$: denoted $f$.

\end{itemize}

\item Domain: The set of a functions arguments. 
\item Range: The set of its values. (Results) 
\item $f$ is a function from a set $A$ iff the domain of $f$ is included in $A$ 

\item $f$ is a function to a set $B$ iff its range is included in $B$. 
%\item $f$ is a total function of a set $A$ iff $f$ assigns a value to each number of $A$.

\item $f^{-1}$ is the inverse of the function $f$ from the set $A$ to the set $B$ iff:if for every member $b \in B$, there is exactly one member of $a \in A$ such that $f(a)=b$, then $f^{-1}(b)=a$, otherwise $f^{-1}(b)$ is undefined.

\item $f$ is onto $B$ iff $B$ is the range of $f$ (Surjective) \\ 
Alt: (Wikipedia) : $\forall y \in Y, \exists x \in X | y=f(x)$
\item $f$ is one-to-one iff $\forall x \forall y (f(x)=f(y)\implies x=y)$ (Injective)
\item $f$ is a bijection iff $f$ is onto and one-to-one.
\item $f$ is a correspondence iff $f$ is total, one-to-one and onto.
\item Sets $A$ and $B$ are equinumerous iff there is a correspondence from $A$ to $B$.% (Isomorphism?-esk)
%Course plan: Review, Do halting problem, compelteness theorem, misc crap.
\end{itemize}


%Lecture 3
\begin{proof}[Equinumerous is transitive]
Prove: if $A$ is equinumerous with $B$ and $B$ is equinumerous wit $C$, then $A$ is equinumerous with $C$.
Proof: Suppose $A$ is equinumerous to $B$, and $B$ is equinumerous to $C$. Then:
There is a total,one-to-one function $f$ from $A$ onto $B$, and a total one-to-one function $g$ from $B$ to $C$.
Prove equinumerous via h=g(f), such that h(n)=g(f(n))
\begin{itemize}
\item h is total: Let $a$ be a member of $A$. $h(a) = g(f(a))$. Since f is total there is a member of $b$ of $B$ such that $f(a)=b)$. since $g$ is total, there is a member of $c \in C$ such that $g(b)=c$. Hence, $h$ is total.
\item $h$ is onto $C$. WLOG Let $c$ be a member of $C$, as $g$ is onto, $\exists b \in B$ such that $g(b)=c$. As $f$ is onto, then $\exists a \in A $ such that $f(a)=b$. Hence, the composition of $h=f(g)$ is onto $C$.
\item $h$ is one-to-one: Suppose $h$ is not one-to-one. \\
Then there $\exists a_1,a_2 \in A$ such that $h(a_1)=h(a_2), a_1 \not = a_2$.\\
Giving $g(f(a_1))=g(f(a_2)), a_1 /not = a_2$\\
Since $g$ is one-to-one $g(b_1) = g(b_2)$ iff $b_1=b_2$. \\
So the issue must lie in $f$.  However $f$ is one-to-one $f(a_1) = f(a_2)$ iff $f(a) = f(b)$.
Which is a contradiction, giving us that $h$ is one-to-one.
\end{itemize}
\end{proof}


%next lecture:

$A^{n}:$ the $n$th Cartesian product of $A$ with itself.

%%Another lecture of pure joy.

%Remark
Suppose that the set of real Numbers $r, r \lt r \lt 1$, is enumerable.
Then $L_r : r_1,r_2,r_3....$ written in a notation of  $0.n_1 n_2 n_3$.($n being natural numbers$)

%Remark
The set of functions form the set of positive integers to positive integers is not enumerable.
\begin{proof}
Suppose S is enumerable.\\
Then there is a list $L_s$ of the members of $S$.\\
$L_s = \{ s_1,s_2,s_3,\cdots \}$\\
Let $\forall n \in \mathbb{N}, n \in k \iff n \not \in S_n$\\
$k$ is a set of positive integers. \\
so There is a number $j$such that $k=s_j$.
So $j \in S_j \iff j \not in S_j$ \\
Hence $S$ is not enumerable. \\
\end{proof}

The set of total nomadic functions from the set of positive integers, $F^{1}$, is not enumerable.

It's a Proof by contradiction. %(I was too lazy to write it down)

%Another lecture.
%We now go into more turing machines stuffs.

Turing machines are in the following form: $q_n, S_{1/0}, S_{1/0}/R/L, q_m$ where $q_n$ is our current state, and you see $S_{1/0}$, perform function $S_{1/0}/R/$ and move to state $q_m$. If there is no operation specified on the current state for a scan, then it halts. (Also Called the Turing Alphabet)

Example with notation:
\begin{tikzpicture}[shorten >=1pt,node distance=2cm,on grid,auto]
% \tilkzstyle{every state}=[f
  \node[state,initial]  (n)                     {$n$};
  \node[state]          (m)    [right of = n]  {$m$};

%    <inital node> edge   node {label} (node_name)
%   (q)   edge        node {}     ()
%  \path[->]    %%TODO uncomment when PDFLATEX starts working to compile this lable
%    (n)  edge                node { $S_{1/0}:(L/R)$ }    (m);  %This is giving me an infinite compile time??
\end{tikzpicture}\\

ex: (These are the same)
$$Q_1S_1RQ_1,Q_1S_0S_1Q_2,Q_2S_1LQ_2,Q_2S_0RQ_3,Q_3S_1S_0Q_3,Q_3S_0RQ_4$$
\begin{tikzpicture}[shorten >=1pt,node distance=2cm,on grid,auto]
% \tilkzstyle{every state}=[f
  \node[state,initial]  (1)                     {$1$};
  \node[state]          (2)    [right of = 1]   {$2$};
  \node[state]          (3)    [right of = 2]   {$3$};
  \node[state]          (4)    [right of = 3]   {$4$};

%    <inital node> edge   node {label} (node_name)
%ex:   (q)   edge        node {}     ()
  \path[->]
    (1)   edge  [loop above]  node    {1:R}   ()
    (1)   edge                node    {0:1}   (2)
    (2)   edge  [loop above]  node    {1:L}   ()
    (2)   edge                node    {0:R}   (3)
    (3)   edge  [loop above]  node    {1:0}   ()
    (3)   edge                node    {0:R}   (4);
\end{tikzpicture}\\

\begin{remark}[Turing Machines]
\begin{itemize}
\item Each Turing machine is a finite set of Turing instructions.
\item Each instruction is a 4 letter word of the Turing Alphabet.
\item The set of Turing machine is enumerable. (Proof: exercise)
%The set of 1 instruction turing machines is enumerable. (Map them to a two dimentional grid with X being inital states, Y being the final state, and the elements inbetween being the 8 combinations of operations of the two states. Then weave!
%For any n the set of n instruction Turing machines is also enumerbable.
%Use Induction!
%Then show the union of all this is enumerable.

%It could also be done with the number of states being enumerable.
\end{itemize}
\end{remark}

%Another lecture begins!
\begin{definition}[Standard inital configuration]
A Turing machine is in a standard Initial configuration $\iff$ 
\begin{itemize}
\item for some positive integer $k$, there are $k$ blocks of $1$'s on the tape. %Either there is just a single block of 1's, or two or three blocks of ones, and so-on
\item separated by a blank, %a single blank 
\item and the rest of the tape is blank. %or empty.
\item the machine is scanning the left-most $1$ on the tape.
\item the machine is in it's lowest numbered state.
\end{itemize}
ex: $\cdots0010110111000\cdots$ is a $SIC$. (if it's in lowest state)
ex: $\cdots00010000\cdots$ is a $SIC$.
\end{definition}


\begin{definition}[Standard final configuration]
A Turing machine is in a standard final configuration $\iff$ 
\begin{itemize}
\item there is a single block of $k$ $1$'s
\item and the rest of the tape is blank. 
\item the machine is scanning the left-most $1$ on the tape.
\end{itemize}
ex: $\cdots00111111000\cdots$ is a $SIC$. (if it's in lowest state)
ex: $\cdots00010000\cdots$ is a $SIC$.
\end{definition}


\begin{definition}[Computes a one-place function $f^1$]
A Turing $M$ computes a one-place function $f^1$: \\
if $M$ is started in a $SIC$ with a single block of $k$ $1$'s and
\begin{itemize}
\item if $f^1$ is defined for the argument $k$, then $M$ eventually halts in a $SFC$
\item or if $f^1$ is not defined for the argument $k$, then either $M$ never halts or it halts in a non-standard final configuration.
\end{itemize}
\end{definition}

\begin{remark}
Every Turing machine computes exactly one function of two arguments. %This is just because we can say it does, from how we defined it earlyer.
\end{remark}

\begin{remark}
For any $n$, each Turing Machine computes exactly one function of $n$ arguments.\\
The set of one-place Turing computable functions is enumerable \\
$\vdots$\\
The set of Turing computable functions is enumerable.
\end{remark}


%%Note: I missed a class, however I've eard there wasn't anything done last class


%Another lecture!
\begin{definition}[The halting problem]
The problem of finding an effective method to determine whether a Turing macine will eventually ahlt or not after it is started with some input.
\end{definition}
\begin{proof}[The halting problem is unsolvable]
Ex: $L_M: M_1,\dots$ \\
$h(m,n) =$ \\ % \left $ &  
%  $\begin{cases}
\quad 1 if $M_m$ eventually ahlts after starting with input $n$ \\
\quad 2 if $M_m$ never halts after starting with input $n$ \\ 
%$  \end{cases}$
The halting problem is solvable $iff$ $h$ is computable.
Show: $h$ is not Turing computable. \\
Let $C$ be a copying machine. \\
Let $F$ be $\frac{1}{2}$ flipper.\\
Suppose $h$ is Turing Computable.\\
Let $H$ be a Turing machine that computes $h$.\\
If $h$ is a Turing computable, then $H$ exists.\\
If $H$ exists, then $D(C-H-F)$ exists. \\
Let $D = M_k$, for some $k$. $M_k \in L_M$.\\
\\
Start $D$ with input $k$.
The C-part of $D$ will produce a copy of $k$,
Then the $H-$part will do its job:
\begin{itemize}
\item If $M_k$ will eventually halt after starting with input $k$, then $H$ will produce output $1$.
\item If $M_k$ will never halt after starting with input $k$, then $H$ will produce output $2$.
\end{itemize}
Then the $F-$part will do it's job.
\begin{itemize}
\item If output from $H$ is 1, $F$ will never halt.
\item If output from $H$ is 2, $F$ will eventually halt.
\end{itemize}
Giving us: \\
\begin{itemize}
\item If $M_k$ will eventually halt after starting with input $k$, then $D$ will never halt ater starting with input $k$. \\
\item If $M_k$ will never halt ater starting with input $k$, then $D$ will eventually halt after starting with input $k$. \\
\end{itemize}
So $M_k$ will halt, after starting with input $k$, $\iff$ $D$ will nver halt after starting with input $k$.\\
Then $M_k$ isn't identical with $D$, which is a contradiction! Hence$D$ doesn't exist. So $H$ does not exists. So $h$ is not turing computable.
\end{proof}


%% MORE LECTURE. YAY FEB 9, 2017.

\begin{proof}[Another halting problem..?]
$L_M: M_1, \cdots$
$L_F: F_1, \cdots$ \\
$g(n) = 1,$ if $f_n(n) = 2$ \\
$g(n) = 2,$ otherwise. \\
$g \not = f_k \forall k$ \\
$h(m,n)= 1 $if $M_m$ eventuall halys after starting with input $n$. \\
$h(m,n)= 2$ if $M_m$ never halts after starting with input $n$. \\
$s(m) = 1$, if $M_m$ eventually halts after starting with input $m$. \\
$s(m) = 2$, if $M_m$ never halts after starting with input $m$. \\
\begin{enumerate}
\item The halting problem is solvable iff $h$ is computable.
\item If $h$ is computable, then $s$ is computable.
\item If $s$ is computable, and $TT$'s is true, then $g$ is computable.
\item $g$ is not Turing computable.
\item Turing Thesis is true (Whatever is not Turing Computable is not computable)
\item The halting problem is not solvable.
\end{enumerate}
\begin{proof}[3.]
Suppose $S$ is computable and $TT$ is true. \\
Then: There is a Turing machine $S*$ that computes s.\\
Suppose that we are to calculate $g(n)$, for some $n$. \\
Start $S*$ with input $n$. \\
\begin{itemize}
\item Case 1: S* eventually halts with output 1\\
We know that $M_n$ will eventually ahlt after it is started with input $n$
Start $M_n$ with input $n$, when it halts, inspect the tape.
\begin{itemize}
\item Case 1.1: Halted in SFC
$f_n(n) = 2$ 
$g(n) = 1$
\item Case 1.2: Halted in non-SFC: $f_n$ is undefined.
$f_n(n) \not=2$
\end{itemize}
\large {
And then Blake broke it: \\
As it's a halting problem to figure out if it's in $SFC$?\\ }
\normalsize
$g(n)=2$
\item Case 2: S* eventually halts with output 2
We know that $M_n$ will never halt after it is started with input $n$. \\
So we know that $f_n$ is undefined for the argument $n$. \\
So we know that $g(n) = 2$ \\
\end{itemize}

\end{proof}
\end{proof}
